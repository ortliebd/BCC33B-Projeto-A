\section{Introdução} \label{sec_introducao}
Linguagens de programação de alto nível foram criadas com a intenção de facilitar a escrita de programas para computadores, buscando uma aproximação com as linguagens naturais, possibilitando ao programador manter o foco principalmente na resolução do problema. 

Contudo, as linguagens de alto nível não podem ser executadas diretamente no computador e precisam de uma codificação na qual o computador as entenda. Essa codificação é conhecida como sistema de numeração binária, formada por zero e um (0 e 1) ou nível de tensão alto e baixo. 

Neste artigo explicaremos o processo de transformação utilizando um programa escrito na linguagem C, alto nível, convertendo-o para \textit{assembly} e em seguida para linguagem de máquina, onde finalmente poderá ser executado.

As linguagens \textit{assembly} representam um nível intermediário de instruções, utilizam um montador para traduzir uma notação simbólica para binário e são escritas para arquiteturas específicas de microprocessadores, portanto faremos uso de um processador MIPS (\textit{Microprocessor without Interlocked Pipeline Stages}) do tipo RISC(\textit{Reduced Instruction Set Computer}) dado a sua simplicidade e seus fins didáticos. 

Também é possível realizar o processo reverso, ou seja, transformar um código binário em \textit{assembly} e de \textit{assembly} para uma linguagem de alto nível. Parte deste trabalho consiste na implementação de um \textit{disassembler} (desmontador) em C responsável por converter o código binário em \textit{assembly}.


